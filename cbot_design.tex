\documentclass[a4paper,12pt]{article}
\begin{document}
\title{cbot design}
\author{Chad Gorshing}
\date{9 Jan 2005}
\maketitle
\tableofcontents
\newpage
\section{Software Specification}

The purpose of this document is to give a detailed technical design document
concerning the architechture of the HTTP spider codenamed 'cbot'.  Throughout
development the compliance to the robots standard~\cite{robotsStandard} is of
an upmost importance. All requirements will try to be meet, as well as all
\emph{highly} suggested specifications.

This project was meant to be a learning project and nothing else.  A way to
learn pattern recognition, the Java programming language, regular expressions,
and database design (among many other reasons).

\section{Modules}

The project is broken up into specific \emph{modules}, I use this term lightly
and the definition in the scope of this application is basically a group of
classes that perform a similar function, it does not mean that a source code
file is a module.

\subsection{Retrieval}

Upon starting the process it is possible to pass a parameter as the starting
point of crawling. If no parameters are specified, then the process will
continue on where it left off \ldots according to information in the database.

\subsection{Robot Specification Validity}

The application is meant to follow the robots.txt~\cite{robotsStandard} standard to the fullest extent. Most importantly 

\subsection{Database Processing}

\subsection{Domain/Site Submission}

\section{Starting}

The system will read all information from the database.  All information is repeatedly queried
so any changes will take an immediate effect.

\begin{thebibliography}{9}
    \bibitem{robotsStandard} http://www.robotstxt.org
\end{thebibliography}
\end{document}
